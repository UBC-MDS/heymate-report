% Options for packages loaded elsewhere
\PassOptionsToPackage{unicode}{hyperref}
\PassOptionsToPackage{hyphens}{url}
\PassOptionsToPackage{dvipsnames,svgnames,x11names}{xcolor}
%
\documentclass[
  11pt,
  a4paper,
  DIV=11,
  numbers=noendperiod]{scrartcl}

\usepackage{amsmath,amssymb}
\usepackage{iftex}
\ifPDFTeX
  \usepackage[T1]{fontenc}
  \usepackage[utf8]{inputenc}
  \usepackage{textcomp} % provide euro and other symbols
\else % if luatex or xetex
  \usepackage{unicode-math}
  \defaultfontfeatures{Scale=MatchLowercase}
  \defaultfontfeatures[\rmfamily]{Ligatures=TeX,Scale=1}
\fi
\usepackage{lmodern}
\ifPDFTeX\else  
    % xetex/luatex font selection
\fi
% Use upquote if available, for straight quotes in verbatim environments
\IfFileExists{upquote.sty}{\usepackage{upquote}}{}
\IfFileExists{microtype.sty}{% use microtype if available
  \usepackage[]{microtype}
  \UseMicrotypeSet[protrusion]{basicmath} % disable protrusion for tt fonts
}{}
\makeatletter
\@ifundefined{KOMAClassName}{% if non-KOMA class
  \IfFileExists{parskip.sty}{%
    \usepackage{parskip}
  }{% else
    \setlength{\parindent}{0pt}
    \setlength{\parskip}{6pt plus 2pt minus 1pt}}
}{% if KOMA class
  \KOMAoptions{parskip=half}}
\makeatother
\usepackage{xcolor}
\usepackage[margin=1in]{geometry}
\setlength{\emergencystretch}{3em} % prevent overfull lines
\setcounter{secnumdepth}{5}
% Make \paragraph and \subparagraph free-standing
\makeatletter
\ifx\paragraph\undefined\else
  \let\oldparagraph\paragraph
  \renewcommand{\paragraph}{
    \@ifstar
      \xxxParagraphStar
      \xxxParagraphNoStar
  }
  \newcommand{\xxxParagraphStar}[1]{\oldparagraph*{#1}\mbox{}}
  \newcommand{\xxxParagraphNoStar}[1]{\oldparagraph{#1}\mbox{}}
\fi
\ifx\subparagraph\undefined\else
  \let\oldsubparagraph\subparagraph
  \renewcommand{\subparagraph}{
    \@ifstar
      \xxxSubParagraphStar
      \xxxSubParagraphNoStar
  }
  \newcommand{\xxxSubParagraphStar}[1]{\oldsubparagraph*{#1}\mbox{}}
  \newcommand{\xxxSubParagraphNoStar}[1]{\oldsubparagraph{#1}\mbox{}}
\fi
\makeatother


\providecommand{\tightlist}{%
  \setlength{\itemsep}{0pt}\setlength{\parskip}{0pt}}\usepackage{longtable,booktabs,array}
\usepackage{calc} % for calculating minipage widths
% Correct order of tables after \paragraph or \subparagraph
\usepackage{etoolbox}
\makeatletter
\patchcmd\longtable{\par}{\if@noskipsec\mbox{}\fi\par}{}{}
\makeatother
% Allow footnotes in longtable head/foot
\IfFileExists{footnotehyper.sty}{\usepackage{footnotehyper}}{\usepackage{footnote}}
\makesavenoteenv{longtable}
\usepackage{graphicx}
\makeatletter
\def\maxwidth{\ifdim\Gin@nat@width>\linewidth\linewidth\else\Gin@nat@width\fi}
\def\maxheight{\ifdim\Gin@nat@height>\textheight\textheight\else\Gin@nat@height\fi}
\makeatother
% Scale images if necessary, so that they will not overflow the page
% margins by default, and it is still possible to overwrite the defaults
% using explicit options in \includegraphics[width, height, ...]{}
\setkeys{Gin}{width=\maxwidth,height=\maxheight,keepaspectratio}
% Set default figure placement to htbp
\makeatletter
\def\fps@figure{htbp}
\makeatother
% definitions for citeproc citations
\NewDocumentCommand\citeproctext{}{}
\NewDocumentCommand\citeproc{mm}{%
  \begingroup\def\citeproctext{#2}\cite{#1}\endgroup}
\makeatletter
 % allow citations to break across lines
 \let\@cite@ofmt\@firstofone
 % avoid brackets around text for \cite:
 \def\@biblabel#1{}
 \def\@cite#1#2{{#1\if@tempswa , #2\fi}}
\makeatother
\newlength{\cslhangindent}
\setlength{\cslhangindent}{1.5em}
\newlength{\csllabelwidth}
\setlength{\csllabelwidth}{3em}
\newenvironment{CSLReferences}[2] % #1 hanging-indent, #2 entry-spacing
 {\begin{list}{}{%
  \setlength{\itemindent}{0pt}
  \setlength{\leftmargin}{0pt}
  \setlength{\parsep}{0pt}
  % turn on hanging indent if param 1 is 1
  \ifodd #1
   \setlength{\leftmargin}{\cslhangindent}
   \setlength{\itemindent}{-1\cslhangindent}
  \fi
  % set entry spacing
  \setlength{\itemsep}{#2\baselineskip}}}
 {\end{list}}
\usepackage{calc}
\newcommand{\CSLBlock}[1]{\hfill\break\parbox[t]{\linewidth}{\strut\ignorespaces#1\strut}}
\newcommand{\CSLLeftMargin}[1]{\parbox[t]{\csllabelwidth}{\strut#1\strut}}
\newcommand{\CSLRightInline}[1]{\parbox[t]{\linewidth - \csllabelwidth}{\strut#1\strut}}
\newcommand{\CSLIndent}[1]{\hspace{\cslhangindent}#1}

\KOMAoption{captions}{tableheading,figureheading}
\makeatletter
\@ifpackageloaded{caption}{}{\usepackage{caption}}
\AtBeginDocument{%
\ifdefined\contentsname
  \renewcommand*\contentsname{Table of contents}
\else
  \newcommand\contentsname{Table of contents}
\fi
\ifdefined\listfigurename
  \renewcommand*\listfigurename{List of Figures}
\else
  \newcommand\listfigurename{List of Figures}
\fi
\ifdefined\listtablename
  \renewcommand*\listtablename{List of Tables}
\else
  \newcommand\listtablename{List of Tables}
\fi
\ifdefined\figurename
  \renewcommand*\figurename{Figure}
\else
  \newcommand\figurename{Figure}
\fi
\ifdefined\tablename
  \renewcommand*\tablename{Table}
\else
  \newcommand\tablename{Table}
\fi
}
\@ifpackageloaded{float}{}{\usepackage{float}}
\floatstyle{ruled}
\@ifundefined{c@chapter}{\newfloat{codelisting}{h}{lop}}{\newfloat{codelisting}{h}{lop}[chapter]}
\floatname{codelisting}{Listing}
\newcommand*\listoflistings{\listof{codelisting}{List of Listings}}
\makeatother
\makeatletter
\makeatother
\makeatletter
\@ifpackageloaded{caption}{}{\usepackage{caption}}
\@ifpackageloaded{subcaption}{}{\usepackage{subcaption}}
\makeatother

\ifLuaTeX
  \usepackage{selnolig}  % disable illegal ligatures
\fi
\usepackage{bookmark}

\IfFileExists{xurl.sty}{\usepackage{xurl}}{} % add URL line breaks if available
\urlstyle{same} % disable monospaced font for URLs
\hypersetup{
  pdftitle={Menu Engineering Supercharged},
  pdfauthor={Hankun Xiao, Yasmin Hassan, Jiaxin Zhang, Zhiwei Zhang},
  colorlinks=true,
  linkcolor={blue},
  filecolor={Maroon},
  citecolor={Blue},
  urlcolor={Blue},
  pdfcreator={LaTeX via pandoc}}


\title{Menu Engineering Supercharged}
\usepackage{etoolbox}
\makeatletter
\providecommand{\subtitle}[1]{% add subtitle to \maketitle
  \apptocmd{\@title}{\par {\large #1 \par}}{}{}
}
\makeatother
\subtitle{MDS Capstone Final Report}
\author{Hankun Xiao, Yasmin Hassan, Jiaxin Zhang, Zhiwei Zhang}
\date{}

\begin{document}
\maketitle

\renewcommand*\contentsname{Table of contents}
{
\hypersetup{linkcolor=}
\setcounter{tocdepth}{2}
\tableofcontents
}

\section{Executive Summary}\label{executive-summary}

This project helps restaurant owners make data-informed menu decisions
by turning raw menu text into structured insights. We applied language
models and scoring techniques to identify popular items and generate
actionable recommendations. The solution is scalable and designed for
real-world application.

\section{Introduction}\label{introduction}

Restaurant owners constantly face the challenge of designing menus that
align with customer preferences and evolving market trends. However,
without access to broader market data, these decisions are often based
on intuition or limited local feedback. This project aims to bridge that
gap by transforming publicly available menu data into structured
insights that can support data-informed menu design.

Our capstone partner heymate! provides an all-in-one business management
system, with the majority of its clients being restaurants. By
structuring and analyzing large-scale menu data, our project enables the
partner to offer greater value to their clients through market-informed
recommendations and operational insights.

Our initial exploration focused on collecting internal datasets provided
by the capstone partner, which contained thousands of raw menu records
from their various clients, covering a range of restaurant types.
However, the internal data alone was insufficient to support robust
recommendation modeling due to its limited scale. To address this, we
researched open-source datasets, such as UberEats menus, which included
richer item descriptions, categories, and ratings (Sakib (2023)). These
datasets helped us understand the variation in menu language, portioning
terms, and restaurant classification.

The main objective of our project was to design a structured pipeline
that can standardize menu data and generate a scoring mechanism to
support recommendation use cases. We built a LLM-based classification
module to extract dish base, flavor, combo status, and standardized
restaurant types, using carefully designed prompts. These structured
outputs were then scored using a popularity metric and visualized for
actionable insights.

\section{Data Pipeline Overview}\label{data-pipeline-overview}

\begin{figure}

\caption{\label{fig-data-pipeline-overview}Data Pipeline Overview}

\centering{

\includegraphics[width=0.8\textwidth,height=\textheight]{../image/final_1_data_pipeline_overview.png}

}

\end{figure}%

The core workflow follows a modular pipeline that integrates data
processing, knowledge base construction, and recommendation generation.

\newpage

\subsection{Model Input Pipeline (Horizontal
Flow)}\label{model-input-pipeline-horizontal-flow}

\begin{enumerate}
\def\labelenumi{\arabic{enumi}.}
\tightlist
\item
  Data Ingestion: We start by collecting raw menu data from both
  internal and external sources, covering a wide range of cuisines, item
  types, and descriptions. For this project, we focus on the Uber Eats
  dataset due to its large volume of menu items and availability of
  popularity metrics.
\item
  Feature Engineering: Under suggestion and domain knowledge from the
  project partner, we identify and select the most informative features
  from the raw data. The goal is to retain fields that influence
  customer decisions and remove those that add noise or redundancy.
\item
  Knowledge Base Construction: We build a knowledge base by scoring each
  menu item's popularity using appearance frequency, rating counts, and
  scores. This serves as the foundation for capturing market trends and
  powering recommendations.
\end{enumerate}

\subsection{Testing and Production Pipeline (Vertical
Flow)}\label{testing-and-production-pipeline-vertical-flow}

\begin{enumerate}
\def\labelenumi{\arabic{enumi}.}
\tightlist
\item
  User Input: The recommender system processes JSON inputs from two
  sources: sampled internal data from the development environment for
  evaluation and tuning, and live user inputs from the production
  environment via the web portal for real-time recommendations.
\item
  Recommendation Module: Based on the specified restaurant type, the
  engine searches the knowledge base, ranks menu items using scoring
  logic, and generates a tailored Recommended Item List.
\item
  Output Visualization: Final recommendations are displayed through a
  user-friendly dashboard, enabling clear interpretation and easy action
  on the results.
\end{enumerate}

\section{Transition from Framework to Data
Product}\label{transition-from-framework-to-data-product}

Having established our framework in earlier stages, we shifted our focus
to building a scalable data product. At its core is a recommendation
engine that suggests top-performing dishes by leveraging aggregated
popularity patterns from similar restaurant types.

\subsection{Data Wrangling Module}\label{data-wrangling-module}

\subsubsection{Exploratory Data
Analysis}\label{exploratory-data-analysis}

To support the development of a robust recommendation engine, we first
conducted exploratory data analysis (EDA) to assess data quality and
structure. The insights informed a series of targeted wrangling tasks
aimed at producing a clean, consistent, and high-quality model input
dataset.

\begin{enumerate}
\def\labelenumi{\arabic{enumi}.}
\tightlist
\item
  \textbf{Data Quality and Inconsistencies:} Both the Uber eats and
  internal datasets presented significant quality issues (shown in
  Table~\ref{tbl-null-entries} and Figure~\ref{fig-null-internal}),
  including missing values, formatting inconsistencies and test records,
  posing a major challenge for generating reliable type-based
  recommendations. To address these, we implemented preprocessing logic
  that performs null handling, duplicate removal, and validation of key
  fields before further processing.
\end{enumerate}

\begin{longtable}[]{@{}lr@{}}
\caption{Null Entries in Uber Eats dataset (Model Input
Data)}\label{tbl-null-entries}\tabularnewline
\toprule\noalign{}
column\_name & null\_count \\
\midrule\noalign{}
\endfirsthead
\toprule\noalign{}
column\_name & null\_count \\
\midrule\noalign{}
\endhead
\bottomrule\noalign{}
\endlastfoot
id & 0 \\
restaurant\_name & 0 \\
score & 1958476 \\
ratings & 1958476 \\
restaurant\_type & 2499 \\
full\_address & 33745 \\
menu\_category & 160 \\
menu\_name & 164 \\
menu\_item\_description & 1452305 \\
price & 160 \\
\end{longtable}

\begin{figure}

\caption{\label{fig-null-internal}Null Entries in internal dataset
(Model Testing Data)}

\centering{

\includegraphics[width=0.6\textwidth,height=\textheight]{../image/final_2_null_internal.png}

}

\end{figure}%

\begin{enumerate}
\def\labelenumi{\arabic{enumi}.}
\setcounter{enumi}{1}
\tightlist
\item
  \textbf{Large-Scale Data Processing:} Even after preprocessing, the
  dataset contained over \textbf{3 million} records, making full
  in-memory processing impractical due to runtime and memory
  constraints. To ensure scalability, we implemented batch processing by
  retrieving indexed chunks and deployed it via a Flask framework with
  multithreaded API calls for concurrent execution.
\end{enumerate}

\begin{figure}

\caption{\label{fig-seafood-fried-rice-var}Seafood Fried Rice Spelling
Variation}

\centering{

\includegraphics[width=0.8\textwidth,height=\textheight]{../image/final-6-seafood_fried_rice_products.png}

}

\end{figure}%

\begin{enumerate}
\def\labelenumi{\arabic{enumi}.}
\setcounter{enumi}{2}
\tightlist
\item
  \textbf{Menu Description Variability:} The biggest challenge was the
  inconsistent naming of similar dishes, for example, ``seafood fried
  rice'' appeared in various forms and languages
  (Figure~\ref{fig-seafood-fried-rice-var}). This hindered similarity
  comparisons and recommendation accuracy. To address this, we used
  OpenAI's GPT API for semantic standardization, enhancing consistency
  and model performance.
\end{enumerate}

\subsubsection{Solution Pipeline}\label{solution-pipeline}

All wrangling tasks were encapsulated in modular Python scripts and
integrated into a unified data pipeline. This pipeline:

\begin{enumerate}
\def\labelenumi{\arabic{enumi}.}
\tightlist
\item
  Cleans and preprocesses both internal and Uber eats datasets
\item
  Applies GPT-based semantic normalization to menu fields
\item
  Feeds standardized data into the recommendation engine
\end{enumerate}

For the model input dataset Figure~\ref{fig-training-pipline}, the
processed data is enriched with popularity metrics and serves as the
engine's core knowledge base.

\begin{figure}

\caption{\label{fig-training-pipline}Solution Pipeline (Model Training)}

\centering{

\includegraphics[width=0.8\textwidth,height=\textheight]{../image/final_3_training_pipline.png}

}

\end{figure}%

For the model testing dataset Figure~\ref{fig-testing-pipline}, the same
logic is applied to ensure full consistency. The input can be either the
cleaned internal restaurant type or the user-selected restaurant type,
both of which are passed through the recommender system to generate
top-rated dish recommendations.

\begin{figure}

\caption{\label{fig-testing-pipline}Solution Pipeline (Model Testing)}

\centering{

\includegraphics[width=0.8\textwidth,height=\textheight]{../image/final_4_testing_pipline.png}

}

\end{figure}%

After reviewing the overall pipeline, we now take a closer look at the
LLM-based cleaning component, which plays a key role in addressing
semantic inconsistencies in menu descriptions and enhancing the quality
of downstream recommendations.

\subsection{Data Cleaning Module}\label{data-cleaning-module}

To ensure that meaningful insights can be extracted from raw menu data,
we first needed to address a major challenge: data inconsistency.

Both the internal dataset provided by the partner and the Uber Eats
dataset exhibit several common challenges, including:

\begin{itemize}
\tightlist
\item
  Mixed languages: Many item names contain both English and non-English
  text.
\item
  Inconsistent restaurant types: Broad categories (e.g., ``American'')
  are mixed with specific terms (e.g., ``wings'', ``sandwich''), making
  aggregation difficult.
\item
  Combo indicators: Expressions such as ``Combo,'' ``Set of Two,'' or
  ``Family Meal'' are used inconsistently, complicating identification.
\item
  Quantity terms: Words like ``6 Pc'' add additional noise and make
  parsing less accurate.
\end{itemize}

In addition, the Uber Eats dataset, while richer in features such as
ratings, rating counts and descriptions, also suffers from missing
values. These missing fields affect the quality of metadata, making it
challenging to ensure consistency and completeness during the cleaning
process.

Without resolving these issues, it would have been impossible to run any
structured analysis across restaurants at scale. To tackle this, we
designed a Data Cleaning Module powered by the GPT API. We developed a
prompt-engineering pipeline that includes two main parts:

\begin{enumerate}
\def\labelenumi{\arabic{enumi}.}
\tightlist
\item
  System prompt: Defines the input and output schema as well as cleaning
  rules, such as how to identify the core dish name, extract up to five
  descriptors, determine whether an item is a combo, and standardize the
  restaurant type.
\item
  User prompt: Feeds batches of raw menu rows into the model in a list
  of dictionary format.
\end{enumerate}

The GPT model returns structured output for each menu item, including:

\begin{itemize}
\tightlist
\item
  \texttt{dish\_base}: core identity (e.g., ``fried rice'').
\item
  \texttt{dish\_flavor}: tags like cooking method or toppings (e.g.,
  {[}``chicken''{]}).
\item
  \texttt{is\_combo}: boolean indicating if the item is a combo.
\item
  \texttt{restaurant\_type\_std}: standardized restaurant type aligned
  with Google Maps Food and Drinks category.
\end{itemize}

To ensure extraction quality and reliability, we iteratively refined our
prompt engineering strategy:

\begin{itemize}
\tightlist
\item
  Required vs.~optional fields: We clearly specified which fields (e.g.,
  item name) must be present, and which are optional (e.g., menu
  description). When optional fields were missing, the model was
  instructed to infer based on other inputs.
\item
  Typing rules: We enforced strict formatting in the output, including
  lowercase, singular, American English spellings, to ensure
  consistency.
\item
  Controlled restaurant type output: We constrained the model to select
  from a fixed list of restaurant types aligned with Google Maps Food
  and Drinks categories.
\item
  Combo identification: We embedded recognition logic for different
  combo indicators, such as ``set of,'' ``combo,'' or ``family meal.''
\item
  Prompt length optimization: Through iterative testing, we shortened
  prompt size while maintaining output quality, helping reduce API
  costs.
\item
  Row indexing: Each row in the batch was assigned a unique index to
  link the model's structured output back to other features like rating
  or score.
\end{itemize}

With this module, we were able to clean inconsistent menu data into a
structured format suitable for downstream analysis and recommendation.

\subsection{Recommendations Algorithm}\label{recommendations-algorithm}

\begin{figure}

\caption{\label{fig-rec-flow}Recommendations Workflow}

\centering{

\includegraphics[width=0.8\textwidth,height=\textheight]{../image/final_5_recommendation_flow.png}

}

\end{figure}%

\subsubsection{Feature Engineering}\label{feature-engineering}

Once the LLM produces a cleaned menu dataset (cleaned\_menu\_mds), we
proceed to feature engineering, enriching this data with relevant
popularity metrics. At this stage, we:

\begin{itemize}
\tightlist
\item
  Join the cleaned Uber Eats menu data with the original Uber Eats
  metadata tables:

  \begin{itemize}
  \tightlist
  \item
    \texttt{Menu\_mds\_sorted} (links dish rows to restaurant IDs)
  \item
    \texttt{Restaurants\_mds} (contains rating counts and scores)
  \end{itemize}
\item
  This join reintroduces three key popularity indicators that had been
  removed during cleaning:

  \begin{itemize}
  \tightlist
  \item
    Number of Ratings (ratings)
  \item
    Average Rating (score)
  \item
    Frequency, calculated using a \texttt{SQL\ COUNT\ (*)\ OVER\ (...)}
    partition grouped by \texttt{dish\_base}, \texttt{dish\_flavor}, and
    \texttt{restaurant\_type\_std} We also:
  \end{itemize}
\item
  Remove entries where \texttt{is\_combo\ ==\ True} to avoid noise in
  popularity estimates
\item
  Eliminate duplicates and ambiguous rows to maintain a clean, reliable
  feature matrix
\end{itemize}

The resulting dataset forms the core knowledge base, a structured,
popularity-aware version of Uber Eats data, on which all recommendations
are based.

\subsection{Normalize Matrix}\label{normalize-matrix}

To ensure that the three indicators (rating counts, rating scores,
frequency) are \textbf{comparable} and \textbf{balanced}, we apply
\textbf{MinMaxScaler} to rescale them to the {[}0, 1{]} range. This is
important because the raw values of frequency or rating count can be
heavily skewed or on different scales. Normalized columns include:

\begin{itemize}
\tightlist
\item
  \texttt{freq\_scaled}
\item
  \texttt{rating\_scaled}
\item
  \texttt{score\_scaled}
\end{itemize}

This step creates a uniform basis for computing the final popularity
score.

\subsection{Compute Popularity Score}\label{compute-popularity-score}

We calculate a weighted popularity score for each dish as follows:

\texttt{popularity\_score\ =\ (0.2\ *\ freq\_scaled\ +\ 0.6\ *\ rating\_scaled\ +\ 0.2\ *\ score\_scale)}

These weights were selected based on exploratory testing and domain
intuition from the partner, prioritizing the number of reviews as a
strong signal of customer engagement, an approach supported by industry
practices/research in similar recommendation systems (Chitalia (2023)
Al-Rubaye and Sukthankar (2020))

\begin{itemize}
\tightlist
\item
  \textbf{60\% rating count:} Measures how many people have reviewed; we
  viewed it as the strongest proxy of broad customer engagement
\item
  \textbf{20\% average score:} captures perceived quality but is often
  biased by low volume
\item
  \textbf{20\% frequency}:** reflects widespread presence on menus but
  doesn't always indicate desirability All computed scores are stored in
  a SQL table: \texttt{cleaned\_menu\_with\_popularity} This becomes the
  \textbf{knowledge base} table used for filtering and recommendations.
\end{itemize}

\subsection{Group and Rank Dishes (Filtering \& Output
Logic)}\label{group-and-rank-dishes-filtering-output-logic}

At recommendation time, the system:

\begin{enumerate}
\def\labelenumi{\arabic{enumi}.}
\tightlist
\item
  Accepts input from a restaurant partner (up to 3 restaurant types)
\item
  Filters the knowledge base (\texttt{cleaned\_menu\_with\_popularity})
  using:

  \begin{itemize}
  \tightlist
  \item
    Exact matches on restaurant\_type\_std
  \item
    Filters out duplicate dishes (e.g., same base/flavor combo for the
    same type)
  \end{itemize}
\item
  For multi-type requests, the popularity scores are averaged across the
  selected types
\item
  Dishes are then ranked based on their average popularity score
\end{enumerate}

Finally, the top N dishes (configurable) are returned as output.

\subsubsection{Clarification on Filtering
Scope}\label{clarification-on-filtering-scope}

Currently, our system supports \textbf{structured filtering only}
specifically, by one to three restaurant\_type\_std values (e.g.,
``pizza restaurant'', ``sandwich shop''). These types are extracted
during LLM cleaning and matched exactly during the filtering process.
This was inspired by best practices in popularity-based recommenders,
which focus on transparency and operational simplicity in early-stage
deployment Sreekala (2020). At this stage, we do \textbf{not} support:

\begin{itemize}
\tightlist
\item
  Keyword-based filtering (e.g., searching for ``pepperoni'')
\item
  Semantic/embedding-based matching (e.g., interpreting ``comfort food''
  or ``something spicy'')
\item
  GPT-powered fuzzy search at recommendation time
\end{itemize}

This was a deliberate choice to prioritize \textbf{transparency,
control,} and \textbf{execution speed} in our MVP. Additionally, our
dataset, though enriched and cleaned, lacked consistent free-text
queries or labelled user intent, which made implementing semantic search
or GPT-based query interpretation less feasible at this stage.

\section{Example Use Case: Recommending for a Pizza
Restaurant}\label{example-use-case-recommending-for-a-pizza-restaurant}

To demonstrate how the recommendation engine works in practice, we
include a use case where a partner restaurant is identified as a
\textbf{``pizza restaurant''}. The system filters all dishes from our
cleaned Uber Eats knowledge base that match this restaurant type and
scores each based on its normalized rating, frequency, and rating count.
The weighted score is then used to rank the items, returning the top 10
most relevant dishes.

As shown in the figure below Figure~\ref{fig-rec-example}, the
top-ranked items for a pizza restaurant include popular classics such
as:

\begin{itemize}
\tightlist
\item
  Pizza (cheese) and Pizza (pepperoni) --- widely rated and highly
  frequent across menus.
\item
  Pizza (meat lover) and Spaghetti (meatball) --- dishes with consistent
  performance.
\item
  Less expected but still relevant items like Papadia (parmesan crust)
  also appear due to strong metric combinations.
\end{itemize}

\begin{figure}

\caption{\label{fig-rec-example}Recommendations For a Pizza
Restaurant''}

\centering{

\includegraphics[width=0.8\textwidth,height=\textheight]{../image/final-7-rec-example.png}

}

\end{figure}%

\subsection{Visualization Demo: Surfacing Actionable
Recommendations}\label{visualization-demo-surfacing-actionable-recommendations}

\begin{figure}

\caption{\label{fig-viz-demo}Visualization Demo''}

\centering{

\includegraphics[width=0.8\textwidth,height=\textheight]{../image/final-8-viz.png}

}

\end{figure}%

To make our recommendation results interpretable and actionable, we
built a two-part visualization:

\subsubsection{CRM System Integration
Mockup}\label{crm-system-integration-mockup}

On the left of Figure~\ref{fig-viz-demo}, we show a conceptual UI for
how this recommendation module could be embedded in the Heymate CRM by
the Heymate engineering team. This allows restaurant managers to easily
access insights from their dashboard, specifically under the
``Recommendation'' tab, alongside other operational menus.

\subsubsection{Recommendation Output
Visualization}\label{recommendation-output-visualization}

On the right of Figure~\ref{fig-viz-demo} is an Altair-based bar chart
showcasing the top dishes sorted by computed popularity score. Key
design features include:

\begin{itemize}
\tightlist
\item
  Top 3 highlights using Heymate brand colours (yellow, red, rose).
\item
  Pop score bars are annotated directly with percentage labels.
\item
  Hover tooltips for dish name and score for ease of exploration.
\item
  Combined label (dish base + flavor) for intuitive comprehension.
\end{itemize}

This visualization not only validates that our scoring pipeline works
but also provides a clear and professional way for stakeholders to
compare and act on dish performance. It is also exportable as a
standalone HTML component and ready for dashboard integration.

\section{How to Use the Data Product}\label{how-to-use-the-data-product}

We designed this data product to support Heymate's internal team and
clients in making data-informed menu decisions.

\subsection{Intended Usage}\label{intended-usage}

Heymate can utilize this tool in two primary ways:

\begin{itemize}
\tightlist
\item
  Internal Use Mode:

  \begin{itemize}
  \tightlist
  \item
    Data Update \& Validation: When a new data source becomes available,
    the internal team can upload and process it through the system to
    evaluate performance with updated inputs.
  \item
    Testing \& Iteration: By inputting internal client restaurant types,
    internal users can assess the quality of dish name cleaning and
    observe how the system generates standardized output. This enables
    iterative refinement and ensures reliability before onboarding new
    clients.
  \end{itemize}
\item
  Client-Facing Deployment: On the frontend, a client can simply select
  their restaurant type (e.g.~``pizza restaurant''), and the system
  returns a list of top recommended dishes based on aggregated
  popularity metrics from similar restaurants. This empowers merchants
  to make data-driven menu decisions aligned with current market trends.
\end{itemize}

\subsection{Strengths of the Data
Product}\label{strengths-of-the-data-product}

\begin{enumerate}
\def\labelenumi{\arabic{enumi}.}
\tightlist
\item
  Automated Cleaning \& Standardization: GPT-powered semantic cleaning
  reduces manual effort in processing noisy menu data.
\item
  Scalability: Batch ingestion with Flask routing and SQL Server
  integration allows seamless future expansion as data volume grows.
\item
  Domain-Specific Recommendations: The system leverages aggregated
  restaurant-type-level patterns, delivering tailored suggestions
  relevant to each merchant.
\item
  Plug-and-Play Interface: Simple inputs (restaurant type) return
  actionable recommendations without requiring technical knowledge.
\end{enumerate}

\section{Data Science Methods}\label{data-science-methods}

We applied many data science techniques in our capstone project and are
highlighting 3 here:

\begin{itemize}
\tightlist
\item
  LLM Integration
\item
  Distributed Deployment
\item
  Materialized View
\end{itemize}

\subsection{LLM Integration}\label{llm-integration}

We leveraged large language models (LLMs) to clean and standardize the
menu data. The raw data was not available for direct use in our
recommendation system due to inconsistencies in formatting, spelling
variations, and the presence of multiple languages. By integrating LLMs,
we were able to extract key information of dish bases and dish flavors,
from item names, categories, and descriptions with high accuracy. Even
in the case when the description is missing, the LLMs can still infer
the information from the item name and category, as well as the context
of the restaurant name.

\subsubsection{Limitation}\label{limitation}

This approach requires payment for API usage. After evaluating
trade-offs between computational power and cost, we chose to use the
\textbf{ChatGPT-4o mini} model.

\subsubsection{Alternative Methods
Considered}\label{alternative-methods-considered}

\begin{itemize}
\tightlist
\item
  Regular Expressions: Effective for extracting structured patterns, but
  not feasible here due to inconsistent formatting and multilingual
  input.
\item
  Custom Deep Learning Model: While potentially powerful, this would
  require labelled training data and significant time and computational
  resources. We don't have such resources within our project scope.
\item
  Locally Deployed LLM: This could reduce long-term costs, but setup and
  maintenance would bring practical challenges within our capstone
  timeline.
\end{itemize}

\subsection{Distributed Deployment}\label{distributed-deployment}

Our project involved processing a large dataset---over 3 million rows
from the Uber Eats dataset. Cleaning this data sequentially would have
taken approximately 5,000 hours, which was not feasible within our
timeline. To solve this, we implemented a distributed deployment
infrastructure to significantly speed up processing. We designed and
deployed an HTTP-triggered function to process data in batches. Each
instance is triggered via a web request, with the batch range passed as
parameters. The system logs the task status at both the start and end of
execution. To monitor tasks internally, we also built a dashboard in
Looker Studio. Initially, we planned to deploy using Azure Functions on
our partner's cloud infrastructure. However, due to security
configuration challenges, we were unable to proceed with this plan. As a
fallback, we deployed the system locally using the Flask framework and
ran up to 20 worker instances concurrently, achieving a 20x speedup.

\subsubsection{Limitation}\label{limitation-1}

The number of concurrent worker instances is limited by the ChatGPT API
rate limit.

\subsubsection{Alternative Approaches
Considered}\label{alternative-approaches-considered}

We evaluated other cloud deployment solutions, such as EC2 from Amazon
Web Services and Google Cloud Functions from the Google Cloud Platform.
However, our partner uses Microsoft Azure, and we prioritized
consistency within that ecosystem. In the future, our partner's
engineering team plans to migrate our local deployment to Azure
Functions, once security configurations are in place.

\subsection{Materialized View}\label{materialized-view}

A Materialized View (also known as a Persistent Derived Table) is a
database optimization technique that stores the result of a query as a
physical table. This allows for much faster data retrieval by avoiding
repeated computation over large datasets. Initially, it took around 6
minutes to generate a restaurant recommendation due to the volume of
data. Such delays are unacceptable in production, especially since the
recommendation module will eventually be integrated into the partner's
CRM system. To optimize query performance, we implemented a materialized
view to cache pre-computed results. This optimization reduced runtime
from 6 minutes to just 3 seconds.

\subsubsection{Limitations}\label{limitations}

\begin{itemize}
\tightlist
\item
  Additional Storage Cost: The materialized view aggregates data from
  the original 3-million-row table, incurring some storage cost.
  However, this is minimal relative to the base data and is not a major
  concern.
\item
  Maintenance and Updates: When new data is ingested into the model
  input pipeline, the materialized view must be refreshed. To handle
  this, we established an automated workflow that updates the
  materialized view upon the successful completion of each data
  ingestion task.
\end{itemize}

\section{Justification Over Other
Products/Interfaces}\label{justification-over-other-productsinterfaces}

There are existing solutions that rely entirely on LLMs to build AI
agents for restaurant recommendations. In these systems, user inputs are
translated into prompts and sent to an LLM, which generates
recommendations based on its internal knowledge.

However, this approach has several clear limitations:

\begin{enumerate}
\def\labelenumi{\arabic{enumi}.}
\tightlist
\item
  LLMs are transformer-based models and are not well-suited for handling
  structured logic or computations involving large-scale tabular data.
\item
  Interpretability is low: these systems often function as black boxes,
  making it difficult to understand or explain how the recommendations
  are generated.
\item
  Lack of real market data: These models typically do not incorporate
  up-to-date or domain-specific datasets. In contrast, our system is
  built on a dataset of over 3 million real menu records, providing a
  much more grounded and data-driven foundation for recommendations.
\end{enumerate}

\section{Conclusions and
Recommendations}\label{conclusions-and-recommendations}

\subsection{What Problem Were We
Solving?}\label{what-problem-were-we-solving}

Heymate! seeks to empower its restaurant partners by offering
data-driven menu recommendations that encourage customer return visits.
However, partners often lack clear insights into which menu items
perform well across the market and why. Our project aimed to close this
gap by designing a popularity-based recommendation system that scores
dishes based on the number of ratings, average rating, and frequency
across menus, offering partners a transparent foundation for data-backed
decision-making.

\subsection{How Does Our Solution Address
It?}\label{how-does-our-solution-address-it}

We developed a full-stack pipeline that:

\begin{itemize}
\tightlist
\item
  Cleans and standardizes restaurant menu data using LLMs,
\item
  Joins the cleaned internal menu with Uber Eats data to enrich it with
  restaurant-level popularity signals,
\item
  Computes a weighted popularity score using MinMaxScaler across three
  metrics,
\item
  Filters and ranks menu items based on restaurant type(s),
\item
  Visualizes the results in an interactive Altair chart for use in
  Heymate's CRM.
\end{itemize}

This product functions well as a minimum viable recommendation engine,
particularly for restaurant partners without access to historical
purchase data. Its transparent logic and simple interface make it
accessible for immediate use and experimentation.

\subsection{Limitations}\label{limitations-1}

While our data product offers a practical solution for transforming raw
menu data into structured insights, several limitations remain:

\begin{itemize}
\tightlist
\item
  \textbf{Popularity Bias:} Our current popularity scoring system relies
  on static metrics, such as item frequency or appearance across menus,
  due to the lack of real transaction data. This makes it a proxy
  measure and may not fully reflect true customer preferences. The score
  could be further improved by incorporating additional quantitative
  features like price, review volume, or order frequency if available in
  the future.
\item
  \textbf{Static Reference:} Our analysis is currently based on static
  snapshots of menu data. Without timestamped records, we are unable to
  capture how item popularity or menu composition evolves. This limits
  our ability to detect trends such as seasonal specials or emerging
  bestsellers.
\item
  \textbf{Hard Filtering:} Restaurant type filtering is an exact match
  only (e.g., ``pizza restaurant''), limiting support for broader or
  fuzzy queries like ``comfort food'' or ``date night meals''. A more
  flexible filtering mechanism could improve the versatility of the
  tool.
\item
  \textbf{Evaluation Challenge:} Due to limited deployment and lack of
  real-time feedback from users, our validation process is based on
  illustrative case studies (e.g., the pizza restaurant example). While
  useful for demonstration, A/B testing or real user feedback would be
  necessary to confirm the real-world effectiveness of the tool.
\item
  \textbf{Scalability Constraint:} Although our local pipeline performs
  well, we haven't deployed it to Azure yet due to resource and security
  constraints. As a result, the system is not yet scalable enough for
  real-time or production use cases.
\item
  \textbf{Lack of Granular Personalization:} While our system
  standardizes restaurant types (e.g., ``Chinese restaurant''), it does
  not yet support finer-grained classification at the sub-type level
  (e.g., ``Szechuan,'' ``Cantonese''). Additionally, the recommender
  ignores important factors such as local popularity, pricing, and
  user-specific preferences. These limitations reduce the system's
  ability to deliver tailored insights for diverse audiences.
\end{itemize}

\subsection{Recommendations for
Heymate}\label{recommendations-for-heymate}

To evolve this prototype into a production-ready recommender, we
recommend:

\begin{enumerate}
\def\labelenumi{\arabic{enumi}.}
\tightlist
\item
  Temporal Tracking: Incorporate timestamped data to uncover seasonal
  patterns and trends.
\item
  Restaurant Feedback Loop: Integrate feedback from partner restaurants
  to refine recommendations over time.
\item
  Semantic Filtering: Support flexible, natural-language queries using
  embeddings or GPT-powered matching.
\item
  Cloud Deployment: Deploy to Azure to support scalability and integrate
  directly with Heymate's CRM system.
\item
  Subtype Extraction: Enhance LLM outputs with more detailed tags (e.g.,
  ``spicy,'' ``gluten-free,'' ``vegan'') for deeper filtering.
\item
  Evaluation Framework: Design structured evaluations using click data,
  client interviews, or business KPIs to measure impact.
\end{enumerate}

Our product lays a solid foundation for \textbf{scalable, transparent},
and \textbf{explainable menu recommendations}. With further iteration,
it can evolve into a dynamic and adaptive system that supports Heymate's
long-term vision for partner success.

\newpage{}

\section*{References}\label{references}
\addcontentsline{toc}{section}{References}

\phantomsection\label{refs}
\begin{CSLReferences}{1}{0}
\bibitem[\citeproctext]{ref-alrubaye2020github}
Al-Rubaye, S., and G. Sukthankar. 2020. {``Popularity-Based Ranking of
GitHub Repositories.''} \emph{IEEE}.
\url{https://ieeexplore.ieee.org/document/9458206}.

\bibitem[\citeproctext]{ref-chitalia2023yelp}
Chitalia, A. 2023. {``Yelp Popularity Score Calculator.''}
\url{https://scholarworks.sjsu.edu/etd_projects/1261}.

\bibitem[\citeproctext]{ref-sakib2023ubereats}
Sakib, Ahmed Shahriar. 2023. {``Uber Eats USA Restaurants and Menus.''}
\url{https://www.kaggle.com/datasets/ahmedshahriarsakib/uber-eats-usa-restaurants-menus}.

\bibitem[\citeproctext]{ref-keshetti2020popularity}
Sreekala, Keshetti. 2020. {``Popularity-Based Recommendation System.''}
\url{https://www.researchgate.net/publication/355773477}.

\end{CSLReferences}




\end{document}
